Q1_ What is Problem Solving? => {
    \defined\: =>
     هي عباره عن أي مشكله معقده نوجد لها حل بخطوات محدده و منظمه الهدف حل المشكله و ترجمة هذا الحلول علي هيئه أكواد برمجيه مع مراعاة كفاءة الأكواد
} 
                            Q1============================================end=>
Q2_ What the Steps of Problem Solving? => {
                                \/*الخطوات باللغة العربيه*\/
        1- فهم المشكلة جيداً وتجميع البيانات عنها
        2- تحديد المدخلات والمخرجات و"حالات الحافة" أي المشكلات المتوقعة وغير المتوقعة
        3- تحديد مجموعة من الخطوات الواضحة والمنظمة التي تهدف إلى حل المشكلة بطريقة خوارزمية
        4- تحليل وتقييم كل خطوة وتحديد ما إذا كانت خطوة صحيحة أم لا
        5- تحديد ما إذا كانت الخطوات التي تم تحديدها هي الأفضل أم لا، إن لم تكن الأفضل نعيد مرحلة تحديد الخطوات للحصول على أفضل كفاءة

                            \/*Steps in English*\/
        1- Understand the problem and gather data.
        2- Define inputs, outputs, and "edge cases" (expected and unexpected).
        3- Develop a clear, structured, algorithmic solution (steps).
        4- Analyze and validate each step
        5- Optimize the solution; if needed, refine the steps for better efficiency.
}
                            Q2============================================end=>
Q3_ what is algorithms? =>{
    \defined\Arabic: => 
    هي ببساطة مجموعة من الخطوات المحددة والمنظمة لحل مشكلة معينة أو إنجاز مهمة محددة.
    
    \defined\English: => 
    An algorithm is simply a set of well-defined steps to solve a problem or accomplish a specific task. 
}                           
                            Q3============================================end=>
Q4_ What ar types of algorithms? => {
 \types\in\Arabic: => 
 1- خوارزميات التصفية (الاختيار): وهي خوارزميات تقوم بتصعيد أو تقليل عدد الخيارات.
 2- خوارزميات التجميع (الجمع): وهي خوارزميات تقوم بجمع البيانات من مصادر متعددة.
 3- خوارزميات التكرار (الدوال): وهي خوارزميات تقوم بتحليل البيانات وتكرارها.
 4- خوارزميات القص (القطع): وهي خوارزميات تقوم بقطع البيانات وتحليلها.
 5- خوارزميات البحث (البحث): وهي خوارزميات تقوم ببحث البيانات وتحليلها.
 6- خوارزميات التصنيف (التصنيف): وهي خوارزميات تقوم بتصنيف البيانات وتحليلها.
 \types\in\English: =>
 1- Filtering algorithms (selection): These algorithms increase or decrease the number of options.
 2- Aggregation algorithms (aggregation): These algorithms collect data from multiple sources.
 3- Recursion algorithms (functions): These algorithms analyze and repeat data.
 4- Cutting algorithms (cutting): These algorithms cut data and analyze it.
 5- Search algorithms (search): These algorithms search and analyze data.
 6- Classification algorithms (classification): These algorithms classify and analyze data.
}                                                 
                            Q4============================================end=>
Q5_ What is types search algorithms? =>{
     
    \types\: => {
        1- Linear Search algorithms: => يتم فيه فحص عناصر القائمة واحدًا تلو الآخر حتى يتم العثور على العنصر المطلوب. 
        2- Binary Search: => يُستخدم في القوائم المُرتبة. يتم فيه تقسيم القائمة إلى نصفين في كل مرة حتى يتم العثور على العنصر , يتطلب قائمة مُرتّبة.
    }
}                            
                            Q5============================================end=>
Q6_ How to evaluate the algorithm? =>{
    \evaluation\: => {
        Complexity: =>{
            1- Time complexity: => يُقاس بكمية الوقت الذي يستغرقه تنفيذ الخوارزمية.
            2- Space complexity: => يُقاس بكمية الذاكرة التي تستخدمها الخوارزمية.
        }
        Q6_Note:=> التقيم يعتمد علي نوع المشكلة و نوع البيانات و نوع الخوارزمية و غيرها من العوامل, فقد يكون الأولوية للوقت علي المساحه و العكس و قد يكون هناك توازن بينهم.
    }
}                           
                            Q6============================================end=>
Q7_ What is the difference between the List and Linked List? =>{                          
    List{
        \definition\Arabic: => هي قائمة من العناصر التي يمكن الوصول إليها بواسطة الفهرسة و يمكن إضافة و إزالة العناصر.
        \definition\English: => A list is a collection of elements that can be accessed by indexing, and elements can be added and removed.
        \code: => {
                List<dataType> nameList = [item1, item2, item3, ...];
                nameList.add(item4);
                nameList.remove(item2);
        }

        \Features: => {
            1- Random access: => يمكن الوصول إلى العناصر بواسطة الفهرسة.
            2- Fixed size: => يمكن تحديد حجم القائمة.
            3- Storage in Memory: => يتم تخزين العناصر في الذاكرة بشكل متجاور. 
            4- Performance when access : => الوصول للعناصر يكون أسرع بسبب وجود فهرسة.
            5- Performance when add or remove: => إضافة أو إزالة العناصر يكون أبطأ بسبب تحديث الفهرسة.
        }
    }
    Linked List {
        \definition\Arabic: => هي قائمة من العناصر التي يمكن الوصول إليها بواسطة العنصر السابق و العنصر التالي و يمكن إضافة و إزالة العناصر.
        \definition\English: => A list of elements that can be accessed by the previous and next element, and elements can be added and removed.
        \code: =>{
            \note\1: (node)  تسمي عناصر القائمة المرتبطة.
            \note\2: (head) هو أول عنصر في القائمة المرتبطة. 
            \note\3: (tail) هو آخر عنصر في القائمة المرتبطة.
            \note\4: (address) لكل عنصر في القائمة المرتبطة يوجد عنوان.
            \note\5: (null) أخر عنصر يشير إلى عدم وجود عنصر تالي. 
            class Node {
                dataType data;
                Node next;
                Node({required this.data, this.next});
            }
            Node head = Node({data: data1, next: null});
            Node node2 = Node({data: data1, next: null});
            head.next = node2;
        }
        \Features: => {
            1- Sequential access: => يتم الوصول إلى العناصر بواسطة العنصر السابق و العنصر التالي .
            2- Dynamic size: => لا يوجد حجم ثابت للقائمة.
            3- Memory allocation: => يتم تخزين العناصر في الذاكرة بشكل منفصل.
            4- Performance when access: => الوصول للعناصر يكون أبطأ بسبب عدم وجود فهرسة.
            5- Performance when add or remove: => إضافة أو إزالة العناصر يكون أسرع بسبب عدم وجود فهرسة.
        }

        Q7_Note: => يتم تحديد نوع القائمة بناءً على نوع البيانات و نوع العمليات المطلوبة, فإذا كانت العمليات الأساسية هي الإضافة و الحذف يفضل استخدام القوائم المرتبطة و إذا كانت العمليات الأساسية هي الوصول يفضل استخدام القوائم. 
    }   

}     
                            Q7============================================end=>