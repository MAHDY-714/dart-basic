Q1_ what is oop?=> {
   \definition:=>  هي نموزج برمجي له مبادئ و أساليب محدده كالتصنيف والتوريث وتعدد الأشكالو التغليف يعتمد علي فكرة تمثيل الكائنات علي هيئة كود و يمكن القول أنه طريقه لتنظيم تفكير المبرمجين في كتابه الأكواد و تنظيم و تصميم البرامج مما يجعلها أكثر مرونة و قابله للتوسع وللتعديل  و الصيانه
    Q1_Key points=> {
    Programming paradigm:- An organized way of writing programs.
    Object-based: Represents everything in a program as an object with properties and behaviors.
        Uses specific principles: classification, inheritance, polymorphism, and encapsulation.
    Aims to: organize code, increase flexibility, facilitate maintenance, and reduce errors.
    } 
    Q1_Steps to create a (OOP) as an idea ? => {
        Step 1: تحديد الفصيل
            تحديد نوع الكائن الذي سوف يتم تمثيله في البرامج مثال الإنسان
                        
        Step 2: تحديد الخصائص و السمات
            تحديد الخصائص و السمات الخاصه بالكائن مثال الاسم و العمر و الجنس و لون البشره والطول و الوزن و غيرها من الخصائص و السمات
    }
}    
                     Q1<==========================================>end=>
Q2_ Steps to create (OOP) code => {
        Step 1: (نوع الكائن)تحديد الفصيله
            class NameClass{
                dataType? attributeName1;
                dataType? attributeName2;
                dataType? methodName(dataType parameterName){//code;}
            }
    
        Step 2:  إنشاء كائن 
            NameClass NameClass = NameClass();
    
        Q2_Note1 => The Variable is attribute in class.
        Q2_Note2 => The Function is method in class.
}
                     Q2<==========================================>end=>
Q3_ What is Constructors => {   
    \definition:=>   إنها دالة خاصة تحمل نفس اسم الفئة وتُستخدم لإنشاء كائنات جديدة من نفس نوع الفئة.
    Q3_Steps to create a default constructor => {
            Step 1: إنشاء داله باسم الفئة
                class NameClass{
                    dataType? attributeName1;
                    late dataType attributeName2;
                    
                    //-constructors-//
                    NameClass(){//code;}
                } 
            Step 2: لإضافة سمات إلى المنشئ، هناك طريقتان
                1- creates new parameters to add their arguments to the attributes=>{
                    NameClass(dataType parameterName1, dataType parameterName2){
                        attributeName1 = parameterName1;
                        attributeName2 = parameterName2;
        
                        \note\1:- This way is used when the name of parameter != name of attribute 
                    }
                                \\----OR----\\
                    NameClass(dataType attributeName1, dataType attributeName2){               
                        this.attributeName1 = attributeName1;     
                        this.attributeName2 = attributeName2;     
        
                        \note\2:- This way is used when the name of parameter == name of attribute 
                    }
        
                }   
                2- uses the attributes as direct parameters to receive their arguments=>{
                    \note\1:- when add code add body\\{}
                    NameClass(this.attributeName1, this.attributeName1){//code}      
                    \\----OR----\\
                    \note\2: when no add code don't add body\\{}
                    NameClass(this.attributeName1, this.attributeName1)
                }
            Q3_Note1 => (this) keyword is used to refer to the current object of the class.
            Q3_Note2 => ({}) These curly braces are used to add code to the body of the constructor and these curly braces are called the body of the constructor.
            Q3_Note3 => (late) is a keyword in dart that means the variable will be initialized later.
    }

    Q3_ Types of Constructors => {
        1.  Default Constructor:   يُستخدم لإنشاء كائنات
        class NameClass{
            NameClass(){//code;}
        }
        
        2. Named Constructor: يُستخدم لإنشاء كائنات جديدة من نفس الفئة ولكن مع بعض الخصائص المحددة 
        class NameClass{
            NameClass.namedContractor({required this.attributeName1, required this.attributeName2});
        }
    }
}                     
                    Q3<==========================================>end=>
Q4_ What is Encapsulation?=> {
    \definition:=>  التغليف هو تجميع البيانات "السمات" والطرق "الوظائف" للتحكم في كيفية الوصول إليها بحيث لا يتم إعطاء قيم غير منطقية أو غير مرغوب فيها، وإخفاء تفاصيل كيفية تنفيذ الوظائف المعقدة "إخفاء التعقيد الداخلي للكائن"، مما يجعل الكود أكثر نمطية وأكثر قابلية للتعديل والصيانة والتطوير

        Q4_Note1 => (_attributeName) When the \attribute is in this form, the \attribute is \private and cannot be accessed outside it is class or the file in which it is referenced.
        Q4_Note2 => (_methodName) When the \method is in this form, the \method is \private and cannot be accessed outside it is class or the file in which it is referenced.
        Q4_Note3 => (setter) The setter is a method that is used to set the value of the attribute.

        Steps to create Encapsulation=> {
            Step 1: إنشاء سمه من نوع الخاص
                dataType? _attributeName;
                Step 2: setter { 
                        \definition: => 
 هي طريقه خاصه بالفئه تتلقى قيمًا خارجية للسمات الخاصة دون إظهار أي تعقيد للكائن و يكون اسمها نفس اسم السمه 
 
                        \note\1: You must add (set) keyword before the attribute Name to make it a setter method
                        \note\2: when call setter attribute in class Preferred to write (this._attributeName) 
                        \code : =>{
                            set attributeName(dataType parameterName){
                                //*and you can add here any code*//
                                _attributeName = parameterName;
                            }
                        }
                        \how\use :=> {
                            1- when use in class => this._attributeName = value;
                            1- when call by object => objectName.attributeName = value;
                        }
                }   
                Step 3: getter { 
                        \definition: =>
    هي طريقه خاصه بالفئه تُستخدم لقراءة واسترجاع القيم فقط للسمه الخاصه دون إظهار تعقيد للكائن ويكون اسمها نفس اسم السمه 
                        \note\1: You must add (get) keyword before the attribute Name to make it a getter method
                        \note\2: when call getter attribute in class Preferred to write (this._attributeName) 
                        \code : =>{
                            dataType get attributeName => _attributeName
                        }
                        \how\use :=> {
                            1- when use in class => this._attributeName;
                            1- when call by object => objectName.attributeName;
                        }
                }  
        }
}
                     Q4<==========================================>end=>
Q5_ What is Inheritance?=> {
    هو عباره  آلية في البرمجة الكائنيه تقوم علي إنشاء فئات تسمي " فئات الإبن" ترث من فئات تسمي "فئات الأب" كل السمات و الطرق مما يساهم في عمليه إعادة استخدام الكود و تنظيمهم  أيضاً يمكن إضافة سمات و طرق للإبن خاصه به و أيضاً يمكن إعادة تعريف السمات والطرق الموروثه للإبن و تغيير قيامها.
        Q5_ Some technical terms in inheritance=>{
            \1- (فئات الأب) => Superclasses أو parent class  
            \2- (فئات الإبن) => Subclasses أو child class
        }
        Q5_ Steps to create Inheritance=>{
            Step 1: create Superclasses{
                class Superclasses{
                    dataType attributeName1;
                    dataType attributeName2;
                    Superclasses({required this.attributeName1, required this.attributeName2});

                    dataType method(){}
                }
            }
            Step 2: create Subclasses{
                \way\1: when call the attribute to Subclasses constructor from Superclasses constructor:=> {
                    class Subclasses extends Superclasses{
                        
                        Subclasses({super.attributeName1, super.attributeName2});

                        dataType method2(){}
                        
                        @override
                        dataType method(){}


                    }
                }
                \way\2: when call the attribute to Subclasses constructor from Superclasses constructor:=> {
                    class Subclasses extends Superclasses{             
                        Subclasses({required dataType attributeName1, required dataType attributeName2}) super(
                            {
                                attributeName1: attributeName1, 
                                attributeName2: attributeName2,
                            });
                    }
                }
            }
        }
    Q5_Note1 => (extends) The keyword used to inherit from a parent class.
    Q5_Note2 => (super) The keyword used to access the parent class's properties and methods.
    Q5_Note3 => (@override) The keyword used to override a parent class's method.
    Q5_Note4 => (super()) The keyword used to call the parent class's constructor.
}
                    Q5<==========================================>end=>
Q6_ What is abstract class in OOP? => {
   \definition: => هو نوع خاص من الفئات في البرمجه الكائنيه ولا يمكن إنشاء كائنات منه مباشرةً و يوفر واجهة للفئات التي ترث منها و يوفر مجموعه من الطرق منها "الطرق المجرده" و "الطرق المُعلنه" و السمات أيضاَ
    Q6_ Some technical terms in abstract class => {
        \1- (الطرق المجرده) => abstract Method  
        \2- (الطرق المُعلنه) => concrete Method
    }

    Q6_ Steps to create abstract class =>{
        Step 1: create abstract Superclasses{
           abstract class Superclasses{
                dataType attributeName1;
                dataType attributeName2;
                Superclasses({required this.attributeName1, required this.attributeName2});

                dataType abstractMethod();

                dataType concreteMethod(){}
           }
        }
        Step 2: create Subclasses{
            \way\1: when use (implements):=> {
            \note\1: => when used \implements\ keyword, the class must implement all attribute & methods in the Superclasses.
            \note\2: => when used \implements\ keyword, don't use \super keyword in the Subclasses constructor you must use \this keyword instead of \super keyword.

                class Subclasses implements Superclasses{
                    
                    Subclasses({this.attributeName1, this.attributeName2});
                    @override
                    dataType attributeName1;

                    @override
                    dataType attributeName2;

                    @override
                    dataType abstractMethod();
                    
                    @override
                    dataType concreteMethod(){//code;}


                }
            }
            \way\2: when use (extends):=> {
            \note\1: => when used \extends\ keyword, the class must implement abstract method in the Superclasses.
            \note\1: => when using the keyword \extends\, you can call concrete methods from superclasses to subclasses and overriding or you can not call them.
                class Subclasses extends Superclasses{             
                    Subclasses({required super.attributeName1, required super.attributeName2});
                    @override
                    dataType abstractMethod();
                    
                    @override
                    dataType concreteMethod(){//code;}
               }
            }
        }
    }
    Q6_ when use abstract class? => {
        \use\1: لو كنا نريد تنفيذ التوريث دون التعامل بشكل مباشر مع الفئات العليا "عدم إنشاء كائن" منه
        \use\2: إنشاء طرق مجرده و الفائده منها أن في المشروعات الكبيره قد يكون عندك تنفيذ لنفس الطريقه بأكثر من شكل و لكن قد تكون نفي المعلمات أو أكثر فستوفر عليك عبئ إنشاءها كل مره و تنظم كودك
    }
    Q6_Note1 => (abstract Method) These are declared methods and do not have a body and must be called in the inherited classes \Subclasses.
    Q6_Note2 => (concrete Method) These are fully defined methods that have a body and must be called or not in the inherited classes \Subclasses In that case they are redefined .
    Q6_Note3 => (abstract) It is a keyword that must be added when creating the \Superclasses.
    inherited classes \Subclasses In that case they are redefined .
}
                     Q6<==========================================>end=>
Q7_ What is mixin? =>{
    \definition: => هو عباره عن مجموعه من "الطرق" و "السمات" يمكن دمجها مع صنف ما دون اللجوء إلي الوراثه بشكل مباشر و يمكن أن نقول عليه مثل "الوراثه المتعدده" في العديد من لفات البرمجه ولكنه يُنفذ بطريقه مختلفه وأكثر أمانًا و هدفه تنظيم الكود وتسهيل قراءته وتعديله و إضافة مهام إضافيه للفئات
    Steps to use mixin => {
        Step1 : use the keyword \mixin\ => {
            mixin MixinClass1 => {
                //code
            }

            mixin MixinClass2 => {
                //code
            }
        }

        Step 2 : call the mixin in the class => {
            class Subclasses extends Superclasses with MixinClass1, MixinClass2 {
            }
        }
    }    
    Q7_Note1 => (mixin) use this keyword before the class name to make it a \MixinClass\.
    Q7_Note2 => (with) use this keyword to call the \MixinClass\ in the subclass.
    Q7_Note3 => you can call more than one \MixinClass\ in the subclass.
}    
                     Q7<==========================================>end=>
